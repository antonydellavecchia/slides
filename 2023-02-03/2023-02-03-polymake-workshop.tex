\documentclass[9pt]{beamer}
\beamertemplatenavigationsymbolsempty
\usetheme{Berlin}

\usepackage{etex, tikz, array, graphics, xspace, relsize, multirow}
\usepackage{ulem}

\input binhex

% For code inclusion
\usepackage{listings}
\lstset{ breaklines=true}
\lstset{escapeinside={<@}{@>}}
\usepackage{algorithm2e}
\usepackage{algorithmic}

% Commands
\newcommand\A{\mathcal{A}}
\newcommand\cc{\mathcal{C}}
\newcommand\codim{\mathrm{codim}}
\newcommand\CP{\mathbb{CP}}
\newcommand\C{\mathbb{C}}
\newcommand\D{\mathrm{D}}
\newcommand\hto{\hookrightarrow}
\newcommand\I{^{-1}}
\newcommand\oo{\mathcal{O}}
\renewcommand\phi{\varphi}
\newcommand\Pj{\mathbb{P}}
\newcommand\pow{\mathcal{P}}
\newcommand\RP{\mathbb{RP}}
\newcommand\rstr[2]{{\left.#1\right|_{#2}}}
\newcommand\R{\mathbb{R}}
\newcommand\V{\mathcal{V}}
\newcommand\F{\mathcal{F}}
\newcommand\G{\mathcal{G}}
\newcommand\Graph{\G(\F_{\mathcal{L}, b}, G)}
\newcommand\set[1]{\{#1\}}
\newcommand\toi{\xrightarrow{\sim}} % category
\newcommand\Z{\mathbb{Z}}


% For drawing
%% Tikz drawing
\usepackage{tikz}
\usetikzlibrary{arrows}
\usetikzlibrary{arrows.meta}
\usepackage{pgfplots}
\usepackage{tikz-3dplot}
\usepackage{tcolorbox}
\usetikzlibrary{matrix,fit,positioning,shapes.geometric,patterns,backgrounds}
\usetikzlibrary{decorations.pathreplacing}
\usetikzlibrary{decorations.markings}
\usetikzlibrary{arrows,calc}
\tikzstyle{bigbox} = [draw=blue!50, thick, rounded corners, rectangle]
\tikzset{
>=stealth'
}
\tikzset{->-/.style={decoration={
  markings,
  mark=at position #1 with {\arrow{>}}},postaction={decorate}}}

%% Software names
\newcommand\topcom{\texttt{TOPCOM}\xspace}
\newcommand\mptopcom{\texttt{MPTOPCOM}\xspace}
\newcommand\mptopcomone{\texttt{MPTOPCOM-1}\xspace}
\newcommand\mts{\texttt{mts}\xspace}
\newcommand\mplrs{\texttt{mplrs}\xspace}
\newcommand\soplex{\texttt{soplex}\xspace}
\newcommand\openmpi{\texttt{OpenMPI}\xspace}
\newcommand\mpi{\texttt{MPI}\xspace}
\newcommand\gfan{\texttt{Gfan}\xspace}
\newcommand\cddlib{\texttt{cddlib}\xspace}
\newcommand\polydb{\texttt{PolyDB}\xspace}
\newcommand\oscar{\texttt{OSCAR}\xspace}

\usepackage{xcolor}
\definecolor{green}{rgb}{0.1,0.59,0.1}
\definecolor{yellow}{rgb}{0.8,0.67,0}
\definecolor{red}{rgb}{0.89,0.1,0.1}
\definecolor{blue}{rgb}{0.1,0.1,0.89}

\usenavigationsymbolstemplate{}

\usepackage{booktabs}
% For software citations
\newcommand{\polymake}{\texttt{po\-ly\-ma\-ke}\xspace}
\newcommand{\polymakejl}{\texttt{Po\-ly\-ma\-ke.jl}\xspace}
\newcommand{\singular}{\texttt{Sin\-gu\-lar}\xspace}
\newcommand\CPP{C\nolinebreak\hspace{-.05em}\raisebox{.4ex}{\relsize{-3}{\textbf{+}}}\nolinebreak\hspace{-.10em}\raisebox{.4ex}{\relsize{-3}{\textbf{+}}}\xspace}

\usepackage{amsmath}
% Newcommands specifically for this article
\newcommand{\eval}{v}               % evaluation function giving switch table
\newcommand{\graph}{\Gamma}         % reverse search graph
\newcommand{\group}{G}              % group acting on point config
\newcommand{\groupElem}{g}          % element of group
\newcommand{\jbound}{\psi}          % bound on the number of elements of set J
\newcommand{\switchTableSize}{\mu}  % index of last non-trivial row in switch table

\newcommand{\pc}{\mathcal P\mathcal C}
\newcommand{\ZZ}{\mathbb Z}
\renewcommand{\AA}{\mathcal A}
\newcommand{\QQ}{\mathbb Q}
\newcommand{\OO}{\mathcal O}
\newcommand{\CC}{\mathbb C}
\newcommand{\PP}{\mathbb P}
\newcommand{\RR}{\mathbb R}
\newcommand{\scalp}[1]{\langle #1 \rangle}
\newcommand{\wt}{\omega}
\newcommand{\cT}{\mathcal T}
\renewcommand{\O}{\mathcal O}
\newcommand{\adm}{\mathcal A(\D, M)}
\newcommand{\blue}[1]{{\usebeamercolor[fg]{palette primary}#1}}

\DeclareMathOperator{\CaDiv}{CaDiv}
\DeclareMathOperator{\conv}{conv}
\DeclareMathOperator{\below}{defect}
\DeclareMathOperator{\vertex}{vertex}
\DeclareMathOperator{\Cox}{Cox}
\DeclareMathOperator{\cl}{cl}
\DeclareMathOperator{\cone}{cone}
\DeclareMathOperator{\Ext}{Ext}
\DeclareMathOperator{\Tor}{Tor}
\DeclareMathOperator{\lcm}{lcm}
\DeclareMathOperator{\Quot}{Quot}
\DeclareMathOperator{\Spec}{Spec}
\DeclareMathOperator{\Sets}{Sets}
\DeclareMathOperator{\relint}{relint}
\DeclareMathOperator{\rk}{rk}
\DeclareMathOperator{\smallestFace}{smallestFace}
\DeclareMathOperator{\Pic}{Pic}
\DeclareMathOperator{\Hom}{Hom}
\DeclareMathOperator{\vol}{vol}
\DeclareMathOperator{\TV}{TV}
\DeclareMathOperator{\tail}{tail}
\DeclareMathOperator{\rep}{rep}
\DeclareMathOperator{\vspan}{span}
\DeclareMathOperator{\canonical}{can}
\DeclareMathOperator{\gkz}{gkz}

\newcommand{\pmsmall}{\includegraphics[scale=0.03]{pmlogo.png}}
\newcommand{\pmlogo}{\includegraphics[scale=0.09]{pmlogo.png}}
\newcommand{\pmbluesmall}{\includegraphics[scale=0.03]{pmbluelogo.png}}
\newcommand{\Disjoint}{\mathop{\coprod}}
\newcommand{\Discriminant}{\mathcal{D}}

\theoremstyle{definition}
\newtheorem{remark}{Remark}

\newtheorem{lem}{Lemma}
\newtheorem{defn}{Definition}

\author{Antony Della Vecchia}
\title{Markov Bases in Polymake}
%\newtheorem*{example}{Example}
\institute[]{
Technische Universit\"at Berlin
}
\date{
2023-02-03
}


\newcommand{\surj}{\twoheadrightarrow}
\newcommand{\oursetting}[1]{\textcolor{blue}{#1}}
\usepackage{listings}
\begin{document}
\maketitle
%%%%%%%%%%%%%%%%%%%%%%%%%%%%%%%%%%%%%%%%%%%%%%%%%%%%%%%%%%%%%%%%%%%%%%%%%%%%%%%
%%%%%%%%%%%%%%%%%%%%%%%%%%%%%%%%%%%%%%%%%%%%%%%%%%%%%%%%%%%%%%%%%%%%%%%%%%%%%%%
%%%%%%%%%%%%%%%%%%%%%%%%%%%%%%%%%%%%%%%%%%%%%%%%%%%%%%%%%%%%%%%%%%%%%%%%%%%%%%%
\begin{frame}[fragile]{Overview}
  \begin{tcolorbox}
    \tableofcontents
  \end{tcolorbox}
\end{frame}
%%%%%%%%%%%%%%%%%%%%%%%%%%%%%%%%%%%%%%%%%%%%%%%%%%%%%%%%%%%%%%%%%%%%%%%%%%%%%%%
%%%%%%%%%%%%%%%%%%%%%%%%%%%%%%%%%%%%%%%%%%%%%%%%%%%%%%%%%%%%%%%%%%%%%%%%%%%%%%%
%%%%%%%%%%%%%%%%%%%%%%%%%%%%%%%%%%%%%%%%%%%%%%%%%%%%%%%%%%%%%%%%%%%%%%%%%%%%%%%

%%%%%%%%%%%%%%%%%%%%%%%%%%%%%%%%%%%%%%%%%%%%%%%%%%%%%%%%%%%%%%%%%%%%%%%%%%%%%%%
\section{Background and Theory}
%%%%%%%%%%%%%%%%%%%%%%%%%%%%%%%%%%%%%%%%%%%%%%%%%%%%%%%%%%%%%%%%%%%%%%%%%%%%%%%

\begin{frame}[fragile]{Markov Basis}
  \begin{defn}
    \begin{itemize}
      \item Given a lattice $\mathcal{L}  \subset \Z^n$ we define
        \begin{align*}
          \mathcal{F}_{\mathcal{L}, b} := \set{z : z = b \text{ mod } \mathcal{L} , z \in \Z^n_+}.
        \end{align*}
      \item For $S \subset \mathcal{L}$ we define
        $\mathcal{G}(\mathcal{F}_{\mathcal{L}, b}, S)$ to be the undirected graph with vertices
        $\mathcal{F}_{\mathcal{L}, b}$ and edges $(u,v)$ for $u - v \in S$ or $v - u \in S$.
      \item We say $S$ is a \emph{generating set} of $\mathcal{F}_{\mathcal{L}, b}$ if
        $\mathcal{G}(\mathcal{F}_{\mathcal{L}, b}, S)$ is connected.
      \item We say $S$ is a \emph{Markov basis} if $S$ is a generating set for $\mathcal{F}_{\mathcal{L}, b}$ for any $b \in \Z^n$.
    \end{itemize}
  \end{defn}
  \begin{remark}
    Note when $\mathcal{L} \cap \Z^n_+ = \set{0}$, $\mathcal{F}_{\mathcal{L}, b}$ is finite for any $b \in \Z^n$.
  \end{remark}
\end{frame}
%%%%%%%%%%%%%%%%%%%%%%%%%%%%%%%%%%%%%%%%%%%%%%%%%%%%%%%%%%%%%%%%%%%%%%%%%%%%%%%
%%%%%%%%%%%%%%%%%%%%%%%%%%%%%%%%%%%%%%%%%%%%%%%%%%%%%%%%%%%%%%%%%%%%%%%%%%%%%%%
%%%%%%%%%%%%%%%%%%%%%%%%%%%%%%%%%%%%%%%%%%%%%%%%%%%%%%%%%%%%%%%%%%%%%%%%%%%%%%%

\begin{frame}[fragile]{Example}
  Let $S = \set{(1, -1, -1, -3, -1, 2), (1, 0, 2, -2, -2, 1)}$, and let $\mathcal{L}$
  be the lattice spanned by $S$, then we have that $\mathcal{L}$ is the integer kernel of
  \begin{align*}
    A =   \begin{bmatrix}
      -2 & -3 & 1 & 0 & 0 & 0\\
      2 & -1 & 0 & 1 & 0 & 0\\
      2 &  1& 0 & 0 & 1 & 0 \\
      -1 & 1 & 0 & 0 & 0 & 1 \\
    \end{bmatrix}
  \end{align*}
  Let $b = (2, 2, 4, 2, 4, 1)$, and consider $\F_{\mathcal{L}, b}$ projected onto the $(x_1, x_2)$-plane.
  \begin{figure}
    \includegraphics[width=0.6\textwidth, height=0.3\textheight]{images/example}
  \end{figure}

\end{frame}
%%%%%%%%%%%%%%%%%%%%%%%%%%%%%%%%%%%%%%%%%%%%%%%%%%%%%%%%%%%%%%%%%%%%%%%%%%%%%%%
%%%%%%%%%%%%%%%%%%%%%%%%%%%%%%%%%%%%%%%%%%%%%%%%%%%%%%%%%%%%%%%%%%%%%%%%%%%%%%%
%%%%%%%%%%%%%%%%%%%%%%%%%%%%%%%%%%%%%%%%%%%%%%%%%%%%%%%%%%%%%%%%%%%%%%%%%%%%%%%


\begin{frame}[fragile]{Lattice Ideals}
  \begin{defn}
    Given a set $S \subset \Z^n$ define the $I(S)$ as
    \begin{align*}
      I(S) := \langle x^a - x^b : a - b \in \mathcal{L} \rangle
      = \langle x^{u_+} - x^{u_-} : u \in S \rangle
    \end{align*}
    When $\mathcal{L} \subset \Z^n$ is a lattice, we call $I(\mathcal{L})$ a
    \emph{lattice ideal}
  \end{defn}
  \begin{lem}
    A set $M$ is a Markov basis if and only if $I(M) = I(\mathcal{L})$
  \end{lem}
\end{frame}
%%%%%%%%%%%%%%%%%%%%%%%%%%%%%%%%%%%%%%%%%%%%%%%%%%%%%%%%%%%%%%%%%%%%%%%%%%%%%%%
%%%%%%%%%%%%%%%%%%%%%%%%%%%%%%%%%%%%%%%%%%%%%%%%%%%%%%%%%%%%%%%%%%%%%%%%%%%%%%%
%%%%%%%%%%%%%%%%%%%%%%%%%%%%%%%%%%%%%%%%%%%%%%%%%%%%%%%%%%%%%%%%%%%%%%%%%%%%%%%

\begin{frame}[fragile]{Term Order and Gr\"obner basis}
  \begin{defn}
    \begin{itemize}
    \item We call $\succ$ a \emph{term order} for $\mathcal{L}$ if $\succ$ is a total
      well ordeing on $\mathcal{F}_{\mathcal{L}, b}$ for all $b \in \Z^n$ and an
      additive order.
    \item We define $\mathcal{L}_{\succ} := \set{u \in \mathcal{L} : u^+ \succ u^-}$.
    \item We call $G \subset \mathcal{L}_{\succ}$ a \emph{Gr\"obner basis} of $\mathcal{L}$
      with respect to $\succ$ if for every $b \in\Z^n_+$ there exists a decreasing path
      in $\mathcal{G}(\mathcal{F}_{\mathcal{L}, b})$ from $b$ to the unique $\succ$-minimal
      element in $\mathcal{F}_{\mathcal{L}, b}$.
    \end{itemize}
  \end{defn}
  \begin{lem}
    Given a term order $\succ$ a set $G \subset \mathcal{L}_{\succ}$ is a
    $\succ$-Gr\"obner basis of $\mathcal{L}$ if and only if for all
    $v \in \mathcal{L}_{\succ}$ there exists $u \in \mathcal{L}_{\succ}$ such
    that $u^+ \leq v^+$, that is, $v^+ - u \in \mathcal{F}_{\mathcal{L}, v^+}$ and
    $v^+ \succ v^+ - u$.
  \end{lem}
\end{frame}
%%%%%%%%%%%%%%%%%%%%%%%%%%%%%%%%%%%%%%%%%%%%%%%%%%%%%%%%%%%%%%%%%%%%%%%%%%%%%%%
%%%%%%%%%%%%%%%%%%%%%%%%%%%%%%%%%%%%%%%%%%%%%%%%%%%%%%%%%%%%%%%%%%%%%%%%%%%%%%%
%%%%%%%%%%%%%%%%%%%%%%%%%%%%%%%%%%%%%%%%%%%%%%%%%%%%%%%%%%%%%%%%%%%%%%%%%%%%%%%

\begin{frame}[fragile]{Reduction Paths}
  \begin{defn}
    A path $(z_0, z_1, \dots, z_i)$ in $\mathcal{G}(\mathcal{F}_{\mathcal{L}, b}, G$ is
    a \emph{reduction path} with respect to $\succ$ if for no $i \in \set{1, \dots, k-1}$
    both $z_i \succ z_0$ and $ z_i \succ z_k$.
  \end{defn}

  \begin{lem}
    Let $\succ$ be a term order of $\Z^n_+$. A set $G \subset \mathcal{L}_{\succ}$
    is a $\succ$-Gr\"obner basis if and only if for any $b \in \Z^n$ and for
    each pair $u, v \in \mathcal{F}_{\mathcal{L}, b}$ there exists a reduction in
    $\mathcal{G}(\mathcal{F}_{\mathcal{L}, b})$ between $u$ and $v$.
  \end{lem}
    \begin{figure}
    \includegraphics[width=0.6\textwidth, height=0.2\textheight]{images/reduction}
  \end{figure}
\end{frame}
%%%%%%%%%%%%%%%%%%%%%%%%%%%%%%%%%%%%%%%%%%%%%%%%%%%%%%%%%%%%%%%%%%%%%%%%%%%%%%%
%%%%%%%%%%%%%%%%%%%%%%%%%%%%%%%%%%%%%%%%%%%%%%%%%%%%%%%%%%%%%%%%%%%%%%%%%%%%%%%
%%%%%%%%%%%%%%%%%%%%%%%%%%%%%%%%%%%%%%%%%%%%%%%%%%%%%%%%%%%%%%%%%%%%%%%%%%%%%%%

\begin{frame}[fragile]{Critical Paths}
  \begin{defn}
    Given $G \subset \mathcal{L}_{\succ}$ and $b \in  \Z^n$, a path  $(\alpha, z, \beta)$
    is a \emph{critical path} if $z \succ \alpha$ and $z \succ\beta$.
  \end{defn}
  \begin{lem}
    A set $G \subset \mathcal{L}_{\succ}$ is a Gr\"obner basis of $\mathcal{L}$ if and only if $G$
    is a generating set of $\mathcal{L}$ and if for all $b  \in \Z^n$ and for every critical
    path $(\alpha, z, \beta)$ in $\Graph$ there exists a reduction path in $\Graph$.
  \end{lem}
  \begin{figure}
    \includegraphics[width=0.2\textwidth, height=0.2\textheight]{images/critical}
    \includegraphics[width=0.4\textwidth, height=0.2\textheight]{images/critical-reduction}
  \end{figure}
\end{frame}
%%%%%%%%%%%%%%%%%%%%%%%%%%%%%%%%%%%%%%%%%%%%%%%%%%%%%%%%%%%%%%%%%%%%%%%%%%%%%%%
%%%%%%%%%%%%%%%%%%%%%%%%%%%%%%%%%%%%%%%%%%%%%%%%%%%%%%%%%%%%%%%%%%%%%%%%%%%%%%%
%%%%%%%%%%%%%%%%%%%%%%%%%%%%%%%%%%%%%%%%%%%%%%%%%%%%%%%%%%%%%%%%%%%%%%%%%%%%%%%

\begin{frame}[fragile]{Minimal Critical Paths}
  \begin{defn}
    \begin{itemize}
    \item A critical path $(\alpha, z, \beta)$ is \emph{minimal} if there does not
      exist another critical path $(\alpha', z', \beta') = (\alpha + \gamma, z + \gamma, \beta + \gamma)$
      for some $\gamma \in \Z^n_+$.
    \item The unique minimal critical path for a pair $(u, v) \in \mathcal{L}$ is denoted by
      $(\alpha^{(u, v)}, z^{(u,v)}, \beta^{(u,v)})$ , where $z^{(u, v)} := max\{u^+, v^+\}$,
      $\alpha^{(u, v)} := z - u$,       $\beta^{(u, v)} := z - v$.
    \end{itemize}
  \end{defn}
  \begin{remark}
    Using minimal critical paths we reduce the amount of reduction paths to be checked to a finite number.
  \end{remark}
\end{frame}
%%%%%%%%%%%%%%%%%%%%%%%%%%%%%%%%%%%%%%%%%%%%%%%%%%%%%%%%%%%%%%%%%%%%%%%%%%%%%%%
%%%%%%%%%%%%%%%%%%%%%%%%%%%%%%%%%%%%%%%%%%%%%%%%%%%%%%%%%%%%%%%%%%%%%%%%%%%%%%%
%%%%%%%%%%%%%%%%%%%%%%%%%%%%%%%%%%%%%%%%%%%%%%%%%%%%%%%%%%%%%%%%%%%%%%%%%%%%%%%

\begin{frame}[fragile]{Projecting and Lifting}
  \begin{lemma}
    \begin{itemize}
    \item If $\mathcal{L} \cap \Z^n_+ \neq 0$, then let $a \in \mathcal{L} \cap \Z^n_+$,
      and we know there is an $i$ such that $a_i > 0$.
      Let $M \subset \mathcal{L}$ be such that $M^i$ is a Markov basis of
      $\mathcal{L}^i$, then $M \cup \{a\}$ is a Markov basis for $\mathcal{L}$
    \item If $\mathcal{L} \cap \Z^n_+ = 0$, then we can find a term order $\succ_c$ on $\mathcal{L}$.
    \item A set $G \subset \mathcal{L}_{\succ_c}$ is a $\succ_c$ Gr\"obner basis for $\mathcal_{\succ_c}$ if and only if $G^i$ is a $\succ_c^i$-Gr\"obner basis for $\mathcal{L}^i$.
    \end{itemize}
  \end{lemma}
\end{frame}
%%%%%%%%%%%%%%%%%%%%%%%%%%%%%%%%%%%%%%%%%%%%%%%%%%%%%%%%%%%%%%%%%%%%%%%%%%%%%%%
%%%%%%%%%%%%%%%%%%%%%%%%%%%%%%%%%%%%%%%%%%%%%%%%%%%%%%%%%%%%%%%%%%%%%%%%%%%%%%%
%%%%%%%%%%%%%%%%%%%%%%%%%%%%%%%%%%%%%%%%%%%%%%%%%%%%%%%%%%%%%%%%%%%%%%%%%%%%%%%

%%%%%%%%%%%%%%%%%%%%%%%%%%%%%%%%%%%%%%%%%%%%%%%%%%%%%%%%%%%%%%%%%%%%%%%%%%%%%%%
\section{Algorithms}
%%%%%%%%%%%%%%%%%%%%%%%%%%%%%%%%%%%%%%%%%%%%%%%%%%%%%%%%%%%%%%%%%%%%%%%%%%%%%%%

\begin{frame}[fragile]{Geometric Buchberger}
  
  \begin{algorithm}[H]
    \textbf{Input:} a term ordering $\succ$ and a set $S \subset \mathcal{L}$\\
    \textbf{Output:} a set $G \subset \mathcal{L}$ such that if $\alpha, \beta$
    are connected in $\G(\F_{\mathcal{L}, \alpha}, G)$ then there exists a reduction
    path between $\alpha$ and $\beta$ in $\G(\F_{\mathcal{L}, \alpha}, G)$.
    \begin{algorithmic}[1]
      \STATE $G \leftarrow \set{u : u^+ \succ u^-, u \in S} \cup \set{u : u^- \succ u^+, u \in S}$ 
      \STATE $C \leftarrow \set{(u,v) : u, v \in G}$
      \WHILE{ $C \neq \emptyset$ }
      \STATE select $(u, v) \in C$
      \STATE $C \leftarrow C \setminus (u,v)$
      \STATE $r \leftarrow MDPA(\alpha^{(u,v)}, G) - MDPA(\beta^{(u,v)}, G)$
      \IF{$r \neq 0$}
      \IF{$r^- \succ r^+$}
      \STATE $r \leftarrow -r$
      \ENDIF
      \STATE $C \leftarrow C \cup \set{(r, s) : s \in G}$
      \STATE $G \leftarrow G \cup r$
      \ENDIF
      \ENDWHILE
      \RETURN $G$
    \end{algorithmic}
  \end{algorithm}

\end{frame}
%%%%%%%%%%%%%%%%%%%%%%%%%%%%%%%%%%%%%%%%%%%%%%%%%%%%%%%%%%%%%%%%%%%%%%%%%%%%%%%
%%%%%%%%%%%%%%%%%%%%%%%%%%%%%%%%%%%%%%%%%%%%%%%%%%%%%%%%%%%%%%%%%%%%%%%%%%%%%%%
%%%%%%%%%%%%%%%%%%%%%%%%%%%%%%%%%%%%%%%%%%%%%%%%%%%%%%%%%%%%%%%%%%%%%%%%%%%%%%%

\begin{frame}[fragile]{Maximal Decreasing Path Algorithm}
  
  \begin{algorithm}[H]
    \textbf{Input:} a vector $\alpha \in \Z^n_+$ and a set $G \subset \mathcal{L}_{\succ}$. \\
    \textbf{Output:} a vector $\alpha'$ where there is a  maximal decreasing path in
    $\G(\F_{\mathcal{L}, \alpha}, G)$ from $\alpha$ to $\alpha'$.
    \begin{algorithmic}[1]
      \STATE $\alpha' \leftarrow \alpha$
      \WHILE{ there is some $u \in G$ such that $u^+ \leq \alpha^+$}
      \STATE $\alpha' \leftarrow \alpha' - u$
      \ENDWHILE
      \RETURN $\alpha'$
    \end{algorithmic}
  \end{algorithm}

\end{frame}
%%%%%%%%%%%%%%%%%%%%%%%%%%%%%%%%%%%%%%%%%%%%%%%%%%%%%%%%%%%%%%%%%%%%%%%%%%%%%%%
%%%%%%%%%%%%%%%%%%%%%%%%%%%%%%%%%%%%%%%%%%%%%%%%%%%%%%%%%%%%%%%%%%%%%%%%%%%%%%%
%%%%%%%%%%%%%%%%%%%%%%%%%%%%%%%%%%%%%%%%%%%%%%%%%%%%%%%%%%%%%%%%%%%%%%%%%%%%%%%
%%%%%%%%%%%%%%%%%%%%%%%%%%%%%%%%%%%%%%%%%%%%%%%%%%%%%%%%%%%%%%%%%%%%%%%%%%%%%%%

\begin{frame}[fragile]{Project and Lift}
  
  \begin{algorithm}[H]
    \textbf{Input:} A lattice $\mathcal{L}$. \\
    \textbf{Output:} A Markov basis $M$ of $\mathcal{L}$.
    \begin{algorithmic}[1]
      \STATE Find a set $\sigma \subset \set{1, 2, \dots, n}$ such that $\text{ker}^{\sigma}(\mathcal{L}) = 0$
      \STATE Find a set $M \subset \mathcal{L}$ such that $M^{\sigma}$ is a Markov basis $\mathcal{L}^i$
      \WHILE{ $\sigma \neq \emptyset$}
      \STATE Select $i \in \sigma$
      \IF{ There exists $a \in \mathcal{L}$ such that $a^{\sigma} \geq 0$ and $a_i > 0$ }
      \STATE $M \leftarrow M \cup \set{ a }$
      \ELSE
      \STATE Find $c \in \R^n_+$ such that $c_{\sigma} = 0$ and $c^{T} u = -u_i$ for all $u \in \mathcal{L}$
      \STATE Using $M$, compute $G \subset \mathcal{L}$ such that $G^{\sigma}$ is a $\succ_c^{\sigma}$ - Gr\"obner basis of $\mathcal{L}^{\sigma}$
      \STATE $M \leftarrow G$
      \ENDIF
      \STATE $\sigma \leftarrow \sigma \setminus \set{i}$
      \ENDWHILE
      \RETURN $M$
    \end{algorithmic}
  \end{algorithm}

\end{frame}


%%%%%%%%%%%%%%%%%%%%%%%%%%%%%%%%%%%%%%%%%%%%%%%%%%%%%%%%%%%%%%%%%%%%%%%%%%%%%%%
%%%%%%%%%%%%%%%%%%%%%%%%%%%%%%%%%%%%%%%%%%%%%%%%%%%%%%%%%%%%%%%%%%%%%%%%%%%%%%%
%%%%%%%%%%%%%%%%%%%%%%%%%%%%%%%%%%%%%%%%%%%%%%%%%%%%%%%%%%%%%%%%%%%%%%%%%%%%%%%

\begin{frame}
  \frametitle{References}
  \footnotesize{
    \begin{thebibliography}{99} % Beamer does not support BibTeX so references must be inserted manually as below
    \bibitem[De Loera, Hemmecke, K\"oppe, 2012]{p1} Jesus DeLoera, Raymond Hemmecke, Mathias K\"oppe, (2013)
    \newblock Algebraic and Geometric Ideas in the Theory of Discrete Optimization
    \newbloack MOS-SIAM Series on Optimization
    \newblock Society for Industrial and Applied Mathematics, Philadelphia, PA

    \bibitem[generating lattice ideal]{p1} Hemmecke, Raymond and Malkin, Peter, (2005)
      \newblock Computing generating sets of lattice ideals
      \newblock arxiv \url{https://arxiv.org/abs/math/0508359}

    \end{thebibliography}
  }
\end{frame}

%%%%%%%%%%%%%%%%%%%%%%%%%%%%%%%%%%%%%%%%%%%%%%%%%%%%%%%%%%%%%%%%%%%%%%%%%%%%%%%
%%%%%%%%%%%%%%%%%%%%%%%%%%%%%%%%%%%%%%%%%%%%%%%%%%%%%%%%%%%%%%%%%%%%%%%%%%%%%%%
%%%%%%%%%%%%%%%%%%%%%%%%%%%%%%%%%%%%%%%%%%%%%%%%%%%%%%%%%%%%%%%%%%%%%%%%%%%%%%%

\end{document}
-