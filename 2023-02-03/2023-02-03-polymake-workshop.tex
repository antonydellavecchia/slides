\documentclass[9pt]{beamer}
\beamertemplatenavigationsymbolsempty
\usetheme{Berlin}

\usepackage{etex, tikz, array, graphics, xspace, relsize, multirow}
\usepackage{ulem}

\input binhex

% For code inclusion
\usepackage{listings}
\lstset{ breaklines=true}
\lstset{escapeinside={<@}{@>}}
\usepackage{algorithm2e}
\usepackage{algorithmic}

% Commands
\newcommand\A{\mathcal{A}}
\newcommand\cc{\mathcal{C}}
\newcommand\codim{\mathrm{codim}}
\newcommand\CP{\mathbb{CP}}
\newcommand\C{\mathbb{C}}
\newcommand\D{\mathrm{D}}
\newcommand\hto{\hookrightarrow}
\newcommand\I{^{-1}}
\newcommand\oo{\mathcal{O}}
\renewcommand\phi{\varphi}
\newcommand\Pj{\mathbb{P}}
\newcommand\pow{\mathcal{P}}
\newcommand\RP{\mathbb{RP}}
\newcommand\rstr[2]{{\left.#1\right|_{#2}}}
\newcommand\R{\mathbb{R}}
\newcommand\V{\mathcal{V}}
\newcommand\H{\mathcal{H}}
\newcommand\set[1]{\{#1\}}
\newcommand\toi{\xrightarrow{\sim}} % category
\newcommand\Z{\mathbb{Z}}


% For drawing
%% Tikz drawing
\usepackage{tikz}
\usetikzlibrary{arrows}
\usetikzlibrary{arrows.meta}
\usepackage{pgfplots}
\usepackage{tikz-3dplot}
\usepackage{tcolorbox}
\usetikzlibrary{matrix,fit,positioning,shapes.geometric,patterns,backgrounds}
\usetikzlibrary{decorations.pathreplacing}
\usetikzlibrary{decorations.markings}
\usetikzlibrary{arrows,calc}
\tikzstyle{bigbox} = [draw=blue!50, thick, rounded corners, rectangle]
\tikzset{
>=stealth'
}
\tikzset{->-/.style={decoration={
  markings,
  mark=at position #1 with {\arrow{>}}},postaction={decorate}}}

%% Software names
\newcommand\topcom{\texttt{TOPCOM}\xspace}
\newcommand\mptopcom{\texttt{MPTOPCOM}\xspace}
\newcommand\mptopcomone{\texttt{MPTOPCOM-1}\xspace}
\newcommand\mts{\texttt{mts}\xspace}
\newcommand\mplrs{\texttt{mplrs}\xspace}
\newcommand\soplex{\texttt{soplex}\xspace}
\newcommand\openmpi{\texttt{OpenMPI}\xspace}
\newcommand\mpi{\texttt{MPI}\xspace}
\newcommand\gfan{\texttt{Gfan}\xspace}
\newcommand\cddlib{\texttt{cddlib}\xspace}
\newcommand\polydb{\texttt{PolyDB}\xspace}
\newcommand\oscar{\texttt{OSCAR}\xspace}

\usepackage{xcolor}
\definecolor{green}{rgb}{0.1,0.59,0.1}
\definecolor{yellow}{rgb}{0.8,0.67,0}
\definecolor{red}{rgb}{0.89,0.1,0.1}
\definecolor{blue}{rgb}{0.1,0.1,0.89}

\usenavigationsymbolstemplate{}

\usepackage{booktabs}
% For software citations
\newcommand{\polymake}{\texttt{po\-ly\-ma\-ke}\xspace}
\newcommand{\polymakejl}{\texttt{Po\-ly\-ma\-ke.jl}\xspace}
\newcommand{\singular}{\texttt{Sin\-gu\-lar}\xspace}
\newcommand\CPP{C\nolinebreak\hspace{-.05em}\raisebox{.4ex}{\relsize{-3}{\textbf{+}}}\nolinebreak\hspace{-.10em}\raisebox{.4ex}{\relsize{-3}{\textbf{+}}}\xspace}

\usepackage{amsmath}
% Newcommands specifically for this article
\newcommand{\eval}{v}               % evaluation function giving switch table
\newcommand{\graph}{\Gamma}         % reverse search graph
\newcommand{\group}{G}              % group acting on point config
\newcommand{\groupElem}{g}          % element of group
\newcommand{\jbound}{\psi}          % bound on the number of elements of set J
\newcommand{\switchTableSize}{\mu}  % index of last non-trivial row in switch table

\newcommand{\pc}{\mathcal P\mathcal C}
\newcommand{\ZZ}{\mathbb Z}
\renewcommand{\AA}{\mathcal A}
\newcommand{\QQ}{\mathbb Q}
\newcommand{\OO}{\mathcal O}
\newcommand{\CC}{\mathbb C}
\newcommand{\PP}{\mathbb P}
\newcommand{\RR}{\mathbb R}
\newcommand{\scalp}[1]{\langle #1 \rangle}
\newcommand{\wt}{\omega}
\newcommand{\cT}{\mathcal T}
\renewcommand{\O}{\mathcal O}
\newcommand{\adm}{\mathcal A(\D, M)}
\newcommand{\blue}[1]{{\usebeamercolor[fg]{palette primary}#1}}

\DeclareMathOperator{\CaDiv}{CaDiv}
\DeclareMathOperator{\conv}{conv}
\DeclareMathOperator{\below}{defect}
\DeclareMathOperator{\vertex}{vertex}
\DeclareMathOperator{\Cox}{Cox}
\DeclareMathOperator{\cl}{cl}
\DeclareMathOperator{\cone}{cone}
\DeclareMathOperator{\Ext}{Ext}
\DeclareMathOperator{\Tor}{Tor}
\DeclareMathOperator{\lcm}{lcm}
\DeclareMathOperator{\Quot}{Quot}
\DeclareMathOperator{\Spec}{Spec}
\DeclareMathOperator{\Sets}{Sets}
\DeclareMathOperator{\relint}{relint}
\DeclareMathOperator{\rk}{rk}
\DeclareMathOperator{\smallestFace}{smallestFace}
\DeclareMathOperator{\Pic}{Pic}
\DeclareMathOperator{\Hom}{Hom}
\DeclareMathOperator{\vol}{vol}
\DeclareMathOperator{\TV}{TV}
\DeclareMathOperator{\tail}{tail}
\DeclareMathOperator{\rep}{rep}
\DeclareMathOperator{\vspan}{span}
\DeclareMathOperator{\canonical}{can}
\DeclareMathOperator{\gkz}{gkz}

\newcommand{\pmsmall}{\includegraphics[scale=0.03]{pmlogo.png}}
\newcommand{\pmlogo}{\includegraphics[scale=0.09]{pmlogo.png}}
\newcommand{\pmbluesmall}{\includegraphics[scale=0.03]{pmbluelogo.png}}
\newcommand{\Disjoint}{\mathop{\coprod}}
\newcommand{\Discriminant}{\mathcal{D}}

\theoremstyle{definition}
\newtheorem{remark}{Remark}

\newtheorem{prop}{Proposition}
\newtheorem{defn}{Definition}

\author{Antony Della Vecchia}
\title{Markov Bases in Polymake}
%\newtheorem*{example}{Example}
\institute[]{
Technische Universit\"at Berlin
}
\date{
2023-02-03
}


\newcommand{\surj}{\twoheadrightarrow}
\newcommand{\oursetting}[1]{\textcolor{blue}{#1}}
\usepackage{listings}
\begin{document}
\maketitle
%%%%%%%%%%%%%%%%%%%%%%%%%%%%%%%%%%%%%%%%%%%%%%%%%%%%%%%%%%%%%%%%%%%%%%%%%%%%%%%
%%%%%%%%%%%%%%%%%%%%%%%%%%%%%%%%%%%%%%%%%%%%%%%%%%%%%%%%%%%%%%%%%%%%%%%%%%%%%%%
%%%%%%%%%%%%%%%%%%%%%%%%%%%%%%%%%%%%%%%%%%%%%%%%%%%%%%%%%%%%%%%%%%%%%%%%%%%%%%%
\begin{frame}[fragile]{Overview}
  \begin{tcolorbox}
    \tableofcontents
  \end{tcolorbox}
\end{frame}
%%%%%%%%%%%%%%%%%%%%%%%%%%%%%%%%%%%%%%%%%%%%%%%%%%%%%%%%%%%%%%%%%%%%%%%%%%%%%%%
%%%%%%%%%%%%%%%%%%%%%%%%%%%%%%%%%%%%%%%%%%%%%%%%%%%%%%%%%%%%%%%%%%%%%%%%%%%%%%%
%%%%%%%%%%%%%%%%%%%%%%%%%%%%%%%%%%%%%%%%%%%%%%%%%%%%%%%%%%%%%%%%%%%%%%%%%%%%%%%

%%%%%%%%%%%%%%%%%%%%%%%%%%%%%%%%%%%%%%%%%%%%%%%%%%%%%%%%%%%%%%%%%%%%%%%%%%%%%%%
\section{Preliminaries}
%%%%%%%%%%%%%%%%%%%%%%%%%%%%%%%%%%%%%%%%%%%%%%%%%%%%%%%%%%%%%%%%%%%%%%%%%%%%%%%

\begin{frame}[fragile]{Hulls}
  blah
\end{frame}
%%%%%%%%%%%%%%%%%%%%%%%%%%%%%%%%%%%%%%%%%%%%%%%%%%%%%%%%%%%%%%%%%%%%%%%%%%%%%%%
%%%%%%%%%%%%%%%%%%%%%%%%%%%%%%%%%%%%%%%%%%%%%%%%%%%%%%%%%%%%%%%%%%%%%%%%%%%%%%%
%%%%%%%%%%%%%%%%%%%%%%%%%%%%%%%%%%%%%%%%%%%%%%%%%%%%%%%%%%%%%%%%%%%%%%%%%%%%%%%

%%%%%%%%%%%%%%%%%%%%%%%%%%%%%%%%%%%%%%%%%%%%%%%%%%%%%%%%%%%%%%%%%%%%%%%%%%%%%%%
\section{Algorithms}
%%%%%%%%%%%%%%%%%%%%%%%%%%%%%%%%%%%%%%%%%%%%%%%%%%%%%%%%%%%%%%%%%%%%%%%%%%%%%%%

\begin{frame}[fragile]{}
  blah
\end{frame}
%%%%%%%%%%%%%%%%%%%%%%%%%%%%%%%%%%%%%%%%%%%%%%%%%%%%%%%%%%%%%%%%%%%%%%%%%%%%%%%
%%%%%%%%%%%%%%%%%%%%%%%%%%%%%%%%%%%%%%%%%%%%%%%%%%%%%%%%%%%%%%%%%%%%%%%%%%%%%%%
%%%%%%%%%%%%%%%%%%%%%%%%%%%%%%%%%%%%%%%%%%%%%%%%%%%%%%%%%%%%%%%%%%%%%%%%%%%%%%%

%%%%%%%%%%%%%%%%%%%%%%%%%%%%%%%%%%%%%%%%%%%%%%%%%%%%%%%%%%%%%%%%%%%%%%%%%%%%%%%
\section{Current State of Convex Hull Computations}
%%%%%%%%%%%%%%%%%%%%%%%%%%%%%%%%%%%%%%%%%%%%%%%%%%%%%%%%%%%%%%%%%%%%%%%%%%%%%%%

\begin{frame}[fragile]{}
  blah
\end{frame}



%%%%%%%%%%%%%%%%%%%%%%%%%%%%%%%%%%%%%%%%%%%%%%%%%%%%%%%%%%%%%%%%%%%%%%%%%%%%%%%
%%%%%%%%%%%%%%%%%%%%%%%%%%%%%%%%%%%%%%%%%%%%%%%%%%%%%%%%%%%%%%%%%%%%%%%%%%%%%%%
%%%%%%%%%%%%%%%%%%%%%%%%%%%%%%%%%%%%%%%%%%%%%%%%%%%%%%%%%%%%%%%%%%%%%%%%%%%%%%%

\begin{frame}[fragile]
  \begin{center}
    Thank you!
  \end{center}
\end{frame}

\begin{frame}
  \frametitle{References}
  \footnotesize{
    \begin{thebibliography}{99} % Beamer does not support BibTeX so references must be inserted manually as below
    \bibitem[Joswig, Thorsten, 2013]{p1} Michael Joswig, Thorsten Theobald, (2013)
      \newblock Polyhedral and algebraic methods in computational geometry
      \newblock London: Springer

    \bibitem[chulls 2017]{p1} Assarf, Benjamin and Gawrilow, Ewgenij and Herr, Katrin and Joswig, Michael and Lorenz, Benjamin and Paffenholz, Andreas and Rehn, Thomas, (2017)
      \newblock Computing convex hulls and counting integer points with \texttt{polymake}
      \newblock Mathematical Programming Computation 9(1), 1--38 



  \end{thebibliography}
  }
\end{frame}

%%%%%%%%%%%%%%%%%%%%%%%%%%%%%%%%%%%%%%%%%%%%%%%%%%%%%%%%%%%%%%%%%%%%%%%%%%%%%%%
%%%%%%%%%%%%%%%%%%%%%%%%%%%%%%%%%%%%%%%%%%%%%%%%%%%%%%%%%%%%%%%%%%%%%%%%%%%%%%%
%%%%%%%%%%%%%%%%%%%%%%%%%%%%%%%%%%%%%%%%%%%%%%%%%%%%%%%%%%%%%%%%%%%%%%%%%%%%%%%

\end{document}
