\documentclass[9pt]{beamer}
\beamertemplatenavigationsymbolsempty
\usetheme{Berlin}

\usepackage{etex, tikz, array, graphics, xspace, relsize, multirow}
\usepackage{ulem}

\input binhex

% For code inclusion
\usepackage{listings}
\lstset{ breaklines=true}
\lstset{escapeinside={<@}{@>}}
\usepackage{algorithm2e}
\usepackage{algorithmic}

% Commands
\newcommand\A{\mathcal{A}}
\newcommand\cc{\mathcal{C}}
\newcommand\codim{\mathrm{codim}}
\newcommand\CP{\mathbb{CP}}
\newcommand\C{\mathbb{C}}
\newcommand\D{\mathrm{D}}
\newcommand\hto{\hookrightarrow}
\newcommand\I{^{-1}}
\newcommand\oo{\mathcal{O}}
\renewcommand\phi{\varphi}
\newcommand\Pj{\mathbb{P}}
\newcommand\pow{\mathcal{P}}
\newcommand\RP{\mathbb{RP}}
\newcommand\rstr[2]{{\left.#1\right|_{#2}}}
\newcommand\R{\mathbb{R}}
\newcommand\V{\mathcal{V}}
\newcommand\F{\mathcal{F}}
\newcommand\G{\mathcal{G}}
\newcommand\Graph{\G(\F_{\mathcal{L}, b}, G)}
\newcommand\set[1]{\{#1\}}
\newcommand\toi{\xrightarrow{\sim}} % category
\newcommand\Z{\mathbb{Z}}

\newtheorem*{prop}{Proposition}
\newtheorem*{conj}{Conjecture}

% For drawing
%% Tikz drawing
\usepackage{tikz}
\usetikzlibrary{arrows}
\usetikzlibrary{arrows.meta}
\usepackage{pgfplots}
\usepackage{tikz-3dplot}
\usepackage{tcolorbox}
\usetikzlibrary{matrix,fit,positioning,shapes.geometric,patterns,backgrounds}
\usetikzlibrary{decorations.pathreplacing}
\usetikzlibrary{decorations.markings}
\usetikzlibrary{arrows,calc}
\tikzstyle{bigbox} = [draw=blue!50, thick, rounded corners, rectangle]
\tikzset{
>=stealth'
}
\tikzset{->-/.style={decoration={
  markings,
  mark=at position #1 with {\arrow{>}}},postaction={decorate}}}

%% Software names
\newcommand\topcom{\texttt{TOPCOM}\xspace}
\newcommand\mptopcom{\texttt{MPTOPCOM}\xspace}
\newcommand\mptopcomone{\texttt{MPTOPCOM-1}\xspace}
\newcommand\mts{\texttt{mts}\xspace}
\newcommand\mplrs{\texttt{mplrs}\xspace}
\newcommand\soplex{\texttt{soplex}\xspace}
\newcommand\openmpi{\texttt{OpenMPI}\xspace}
\newcommand\mpi{\texttt{MPI}\xspace}
\newcommand\gfan{\texttt{Gfan}\xspace}
\newcommand\cddlib{\texttt{cddlib}\xspace}
\newcommand\polydb{\texttt{PolyDB}\xspace}
\newcommand\oscar{\texttt{OSCAR}\xspace}

\usepackage{xcolor}
\definecolor{green}{rgb}{0.1,0.59,0.1}
\definecolor{yellow}{rgb}{0.8,0.67,0}
\definecolor{red}{rgb}{0.89,0.1,0.1}
\definecolor{blue}{rgb}{0.1,0.1,0.89}

\usenavigationsymbolstemplate{}

\usepackage{booktabs}
% For software citations
\newcommand{\polymake}{\texttt{po\-ly\-ma\-ke}\xspace}
\newcommand{\polymakejl}{\texttt{Po\-ly\-ma\-ke.jl}\xspace}
\newcommand{\singular}{\texttt{Sin\-gu\-lar}\xspace}
\newcommand\CPP{C\nolinebreak\hspace{-.05em}\raisebox{.4ex}{\relsize{-3}{\textbf{+}}}\nolinebreak\hspace{-.10em}\raisebox{.4ex}{\relsize{-3}{\textbf{+}}}\xspace}

\usepackage{amsmath}
% Newcommands specifically for this article
\newcommand{\eval}{v}               % evaluation function giving switch table
\newcommand{\graph}{\Gamma}         % reverse search graph
\newcommand{\group}{G}              % group acting on point config
\newcommand{\groupElem}{g}          % element of group
\newcommand{\jbound}{\psi}          % bound on the number of elements of set J
\newcommand{\switchTableSize}{\mu}  % index of last non-trivial row in switch table

\newcommand{\pc}{\mathcal P\mathcal C}
\newcommand{\ZZ}{\mathbb Z}
\renewcommand{\AA}{\mathcal A}
\newcommand{\QQ}{\mathbb Q}
\newcommand{\OO}{\mathcal O}
\newcommand{\CC}{\mathbb C}
\newcommand{\PP}{\mathbb P}
\newcommand{\RR}{\mathbb R}
\newcommand{\scalp}[1]{\langle #1 \rangle}
\newcommand{\wt}{\omega}
\newcommand{\cT}{\mathcal T}
\renewcommand{\O}{\mathcal O}
\newcommand{\adm}{\mathcal A(\D, M)}
\newcommand{\blue}[1]{{\usebeamercolor[fg]{palette primary}#1}}

\DeclareMathOperator{\CaDiv}{CaDiv}
\DeclareMathOperator{\conv}{conv}
\DeclareMathOperator{\below}{defect}
\DeclareMathOperator{\vertex}{vertex}
\DeclareMathOperator{\Cox}{Cox}
\DeclareMathOperator{\cl}{cl}
\DeclareMathOperator{\cone}{cone}
\DeclareMathOperator{\Ext}{Ext}
\DeclareMathOperator{\Tor}{Tor}
\DeclareMathOperator{\lcm}{lcm}
\DeclareMathOperator{\Quot}{Quot}
\DeclareMathOperator{\Spec}{Spec}
\DeclareMathOperator{\Sets}{Sets}
\DeclareMathOperator{\relint}{relint}
\DeclareMathOperator{\rk}{rk}
\DeclareMathOperator{\smallestFace}{smallestFace}
\DeclareMathOperator{\Pic}{Pic}
\DeclareMathOperator{\Hom}{Hom}
\DeclareMathOperator{\vol}{vol}
\DeclareMathOperator{\TV}{TV}
\DeclareMathOperator{\tail}{tail}
\DeclareMathOperator{\rep}{rep}
\DeclareMathOperator{\vspan}{span}
\DeclareMathOperator{\canonical}{can}
\DeclareMathOperator{\gkz}{gkz}

\newcommand{\pmsmall}{\includegraphics[scale=0.03]{pmlogo.png}}
\newcommand{\pmlogo}{\includegraphics[scale=0.09]{pmlogo.png}}
\newcommand{\pmbluesmall}{\includegraphics[scale=0.03]{pmbluelogo.png}}
\newcommand{\Disjoint}{\mathop{\coprod}}
\newcommand{\Discriminant}{\mathcal{D}}

\theoremstyle{definition}
\newtheorem{remark}{Remark}

\newtheorem{lem}{Lemma}
\newtheorem{defn}{Definition}

\author{Antony Della Vecchia}
\title{}

\subtitle{Research Data in Discrete Mathematics}
%\newtheorem*{example}{Example}
\institute[]{
Technische Universit\"at Berlin
}
\date{
2023-03-22
}



\newcommand{\surj}{\twoheadrightarrow}
\newcommand{\oursetting}[1]{\textcolor{blue}{#1}}
\usepackage{listings}
\begin{document}

%%%%%%%%%%%%%%%%%%%%%%%%%%%%%%%%%%%%%%%%%%%%%%%%%%%%%%%%%%%%%%%%%%%%%%%%%%%%%%%
%%%%%%%%%%%%%%%%%%%%%%%%%%%%%%%%%%%%%%%%%%%%%%%%%%%%%%%%%%%%%%%%%%%%%%%%%%%%%%%
%%%%%%%%%%%%%%%%%%%%%%%%%%%%%%%%%%%%%%%%%%%%%%%%%%%%%%%%%%%%%%%%%%%%%%%%%%%%%%%


\begin{frame}[fragile]{Universally Free Numerical Semigroups}
  
  \begin{itemize}
  \item $ A = \set{a_1, \dots, a_n} \subset \ZZ^+$  , $\mathcal{S} = \set{u_1a_1 + \dots + u_n a_n \mid u_i \in \mathbb{N}}$.
  \item $\text{deg}_A(\bold{u}) = u_1a_1 + \dots + u_n a_n$.
  \item $A$-homogeneous prime ideals of height $n-1$, $I_A = \langle x^u - x^v \mid deg_A(x^u) = deg_A(x^v) \rangle$.

  \item $A' = \set{a_1 / \text{gcd}(A), \dots, a_1 / \text{gcd}(A) }$, $I_A' = I_A$.
  \item $\mathcal(S) = \langle A' \rangle$ is a \emph{numerical semigroup}.
  \item Looks at \emph{universally free numerical semigroup} and study it's implications.
    
    
  \end{itemize}
\end{frame}

%%%%%%%%%%%%%%%%%%%%%%%%%%%%%%%%%%%%%%%%%%%%%%%%%%%%%%%%%%%%%%%%%%%%%%%%%%%%%%%
%%%%%%%%%%%%%%%%%%%%%%%%%%%%%%%%%%%%%%%%%%%%%%%%%%%%%%%%%%%%%%%%%%%%%%%%%%%%%%%
%%%%%%%%%%%%%%%%%%%%%%%%%%%%%%%%%%%%%%%%%%%%%%%%%%%%%%%%%%%%%%%%%%%%%%%%%%%%%%%

\begin{frame}[fragile]{Universally Free Numerical Semigroups}
  
  \begin{itemize}
  \item $A$ is the gluing of partitions $A_1, A_2$ if $\text{lcm}(\text{gcd}(A_1), \text{gcd}(A_2)) \in \langle A_1 \rangle \cup \langle A_2 \rangle$
  \item $\mathcal{S}$ is free for the arrangement $\set{a_1, \dots, a_n}$ if for each $\set{i, \dots, n}$ the set $\set{a_i, \dots a_n}$ is the gluing of $\set{a_i}$ with $\set{a_{i+1}, \dots a_n}$
  \item $\mathcal{C}_A$ set of circuits of $I_A$
  \end{itemize}

  \begin{prop}
    Let $S$ be a universally free numerical semigroup then $I_A$ is generated by $n-1$ circuits. And some other more technical properties, involving Betti degree and critical binomials.
  \end{prop}

\end{frame}

%%%%%%%%%%%%%%%%%%%%%%%%%%%%%%%%%%%%%%%%%%%%%%%%%%%%%%%%%%%%%%%%%%%%%%%%%%%%%%%
%%%%%%%%%%%%%%%%%%%%%%%%%%%%%%%%%%%%%%%%%%%%%%%%%%%%%%%%%%%%%%%%%%%%%%%%%%%%%%%
%%%%%%%%%%%%%%%%%%%%%%%%%%%%%%%%%%%%%%%%%%%%%%%%%%%%%%%%%%%%%%%%%%%%%%%%%%%%%%%



\begin{frame}[fragile]{Universally Free Numerical Semigroups}
  \begin{prop}
    If $\mathcal{S}$ is a universally free numerical semigroup then $\mathcal{C}_A \subset \mathcal{M}_A$
  \end{prop}

  \begin{conj}
     $\mathcal{S}$ is a universally free numerical semigroup if and only if $\mathcal{C}_A \subset \mathcal{M}_A$. Large 15 day computation suggesting true.
   \end{conj}

   \begin{theorem}
     Let $\mathcal{S}$ be a numerical semigroup, $\mathcal{S}$ is universally free if and only if every reduced Gr\"obner basis of $I_A$ has $n-1$ elements.
   \end{theorem}
 \end{frame}

 %%%%%%%%%%%%%%%%%%%%%%%%%%%%%%%%%%%%%%%%%%%%%%%%%%%%%%%%%%%%%%%%%%%%%%%%%%%%%%%
%%%%%%%%%%%%%%%%%%%%%%%%%%%%%%%%%%%%%%%%%%%%%%%%%%%%%%%%%%%%%%%%%%%%%%%%%%%%%%%
%%%%%%%%%%%%%%%%%%%%%%%%%%%%%%%%%%%%%%%%%%%%%%%%%%%%%%%%%%%%%%%%%%%%%%%%%%%%%%%

 \begin{frame}[fragile]{Universally Free Numerical Semigroups}
   \begin{itemize}
   \item $A$ degrees of any Markov basis are invariant and called the \emph{Betti} degrees of $I_A$.
   \item If $A$ is a finite set of positive integers, $\mathcal{C}_A \subset \mathcal{U}_A \subset Gr_A$ and $Cr_A \subset \mathcal{M}_A \subset Gr_A$.
   \item $A$ is Betti divisible if its Betti degrees are ordered by divisibility.
   \item If $\mathcal{S}$ is a Betti divisible numerical semigroup minimally generated by $A$ then $\mathcal{C}_A = \mathcal{U}_A \subset Cr_A = \mathcal{M}_A = Gr_A$.
   \item Gives characterization of 3 generated universally free numerical semigroups.
   \end{itemize}


 \end{frame}

\end{document}
