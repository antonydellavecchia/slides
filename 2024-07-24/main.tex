\documentclass[9pt]{beamer}
\beamertemplatenavigationsymbolsempty
\usetheme{Berlin}

\usepackage{etex, tikz, array, graphics, xspace, relsize, multirow}
\usepackage{ulem}

\input binhex

% For code inclusion
\usepackage{listings}
\lstset{ breaklines=true}
\lstset{escapeinside={<@}{@>}}
\usepackage{algorithm2e}
\usepackage{algorithmic}

% Commands
\newcommand\A{\mathcal{A}}
\newcommand\cc{\mathcal{C}}
\newcommand\codim{\mathrm{codim}}
\newcommand\CP{\mathbb{CP}}
\newcommand\C{\mathbb{C}}
\newcommand\D{\mathrm{D}}
\newcommand\hto{\hookrightarrow}
\newcommand\I{^{-1}}
\newcommand\oo{\mathcal{O}}
\renewcommand\phi{\varphi}
\newcommand\Pj{\mathbb{P}}
\newcommand\pow{\mathcal{P}}
\newcommand\RP{\mathbb{RP}}
\newcommand\rstr[2]{{\left.#1\right|_{#2}}}
\newcommand\R{\mathbb{R}}
\newcommand\V{\mathcal{V}}
\newcommand\F{\mathcal{F}}
\newcommand\G{\mathcal{G}}
\newcommand\Graph{\G(\F_{\mathcal{L}, b}, G)}
\newcommand\set[1]{\{#1\}}
\newcommand\toi{\xrightarrow{\sim}} % category
\newcommand\Z{\mathbb{Z}}


% For drawing
%% Tikz drawing
\usepackage{tikz}
\usetikzlibrary{arrows}
\usetikzlibrary{arrows.meta}
\usepackage{pgfplots}
\usepackage{tikz-3dplot}
\usepackage{tcolorbox}
\usetikzlibrary{matrix,fit,positioning,shapes.geometric,patterns,backgrounds}
\usetikzlibrary{decorations.pathreplacing}
\usetikzlibrary{decorations.markings}
\usetikzlibrary{arrows,calc}
\tikzstyle{bigbox} = [draw=blue!50, thick, rounded corners, rectangle]
\tikzset{
  >=stealth'
}
\tikzset{->-/.style={decoration={
      markings,
      mark=at position #1 with {\arrow{>}}},postaction={decorate}}}

%% Software names
\newcommand\topcom{\texttt{TOPCOM}\xspace}
\newcommand\mptopcom{\texttt{MPTOPCOM}\xspace}
\newcommand\mptopcomone{\texttt{MPTOPCOM-1}\xspace}
\newcommand\mts{\texttt{mts}\xspace}
\newcommand\mplrs{\texttt{mplrs}\xspace}
\newcommand\soplex{\texttt{soplex}\xspace}
\newcommand\openmpi{\texttt{OpenMPI}\xspace}
\newcommand\mpi{\texttt{MPI}\xspace}
\newcommand\gfan{\texttt{Gfan}\xspace}
\newcommand\cddlib{\texttt{cddlib}\xspace}
\newcommand\polydb{\texttt{PolyDB}\xspace}
\newcommand\OSCAR{\texttt{OSCAR}\xspace}
\newcommand\Julia{\texttt{Julia}\xspace}

\usepackage{xcolor}
\definecolor{green}{rgb}{0.1,0.59,0.1}
\definecolor{yellow}{rgb}{0.8,0.67,0}
\definecolor{red}{rgb}{0.89,0.1,0.1}
\definecolor{blue}{rgb}{0.1,0.1,0.89}

\usenavigationsymbolstemplate{}

\usepackage{booktabs}
% For software citations
\newcommand{\polymake}{\texttt{po\-ly\-ma\-ke}\xspace}
\newcommand{\polymakejl}{\texttt{Po\-ly\-ma\-ke.jl}\xspace}
\newcommand{\singular}{\texttt{Sin\-gu\-lar}\xspace}
\newcommand\CPP{C\nolinebreak\hspace{-.05em}\raisebox{.4ex}{\relsize{-3}{\textbf{+}}}\nolinebreak\hspace{-.10em}\raisebox{.4ex}{\relsize{-3}{\textbf{+}}}\xspace}

\usepackage{amsmath}
% Newcommands specifically for this article
\newcommand{\eval}{v}               % evaluation function giving switch table
\newcommand{\graph}{\Gamma}         % reverse search graph
\newcommand{\group}{G}              % group acting on point config
\newcommand{\groupElem}{g}          % element of group
\newcommand{\jbound}{\psi}          % bound on the number of elements of set J
\newcommand{\switchTableSize}{\mu}  % index of last non-trivial row in switch table

\newcommand{\pc}{\mathcal P\mathcal C}
\newcommand{\ZZ}{\mathbb Z}
\renewcommand{\AA}{\mathcal A}
\newcommand{\QQ}{\mathbb Q}
\newcommand{\OO}{\mathcal O}
\newcommand{\CC}{\mathbb C}
\newcommand{\PP}{\mathbb P}
\newcommand{\RR}{\mathbb R}
\newcommand{\scalp}[1]{\langle #1 \rangle}
\newcommand{\wt}{\omega}
\newcommand{\cT}{\mathcal T}
\renewcommand{\O}{\mathcal O}
\newcommand{\adm}{\mathcal A(\D, M)}
\newcommand{\blue}[1]{{\usebeamercolor[fg]{palette primary}#1}}

\DeclareMathOperator{\CaDiv}{CaDiv}
\DeclareMathOperator{\conv}{conv}
\DeclareMathOperator{\below}{defect}
\DeclareMathOperator{\vertex}{vertex}
\DeclareMathOperator{\Cox}{Cox}
\DeclareMathOperator{\cl}{cl}
\DeclareMathOperator{\cone}{cone}
\DeclareMathOperator{\Ext}{Ext}
\DeclareMathOperator{\Tor}{Tor}
\DeclareMathOperator{\lcm}{lcm}
\DeclareMathOperator{\Quot}{Quot}
\DeclareMathOperator{\Spec}{Spec}
\DeclareMathOperator{\Sets}{Sets}
\DeclareMathOperator{\relint}{relint}
\DeclareMathOperator{\rk}{rk}
\DeclareMathOperator{\smallestFace}{smallestFace}
\DeclareMathOperator{\Pic}{Pic}
\DeclareMathOperator{\Hom}{Hom}
\DeclareMathOperator{\vol}{vol}
\DeclareMathOperator{\TV}{TV}
\DeclareMathOperator{\tail}{tail}
\DeclareMathOperator{\rep}{rep}
\DeclareMathOperator{\vspan}{span}
\DeclareMathOperator{\canonical}{can}
\DeclareMathOperator{\gkz}{gkz}

\newcommand{\pmsmall}{\includegraphics[scale=0.03]{pmlogo.png}}
\newcommand{\pmlogo}{\includegraphics[scale=0.09]{pmlogo.png}}
\newcommand{\pmbluesmall}{\includegraphics[scale=0.03]{pmbluelogo.png}}
\newcommand{\Disjoint}{\mathop{\coprod}}
\newcommand{\Discriminant}{\mathcal{D}}

\theoremstyle{definition}
\newtheorem{remark}{Remark}

\newtheorem{lem}{Lemma}
\newtheorem{defn}{Definition}

\author[Antony Della Vecchia]{Antony Della Vecchia \\ \vspace{4mm} \small{Joint work with M. Joswig and B. Lorenz} }
\title{A FAIR File Format for Mathematical Software}

\institute[]{
  TU Berlin
}
\date{
  ICMS Mathematical Research Data
  2024-07-24
}
\logo{
  \includegraphics[width=2cm]{images/mardi-logo.png}
}


\newcommand{\surj}{\twoheadrightarrow}
\newcommand{\oursetting}[1]{\textcolor{blue}{#1}}
\usepackage{listings}
\begin{document}
\maketitle
%%%%%%%%%%%%%%%%%%%%%%%%%%%%%%%%%%%%%%%%%%%%%%%%%%%%%%%%%%%%%%%%%%%%%%%%%%%%%%% 
%%%%%%%%%%%%%%%%%%%%%%%%%%%%%%%%%%%%%%%%%%%%%%%%%%%%%%%%%%%%%%%%%%%%%%%%%%%%%%% 
%%%%%%%%%%%%%%%%%%%%%%%%%%%%%%%%%%%%%%%%%%%%%%%%%%%%%%%%%%%%%%%%%%%%%%%%%%%%%%% 

\begin{frame}[fragile]{The Importance of Files for Computer Algebra}
    \begin{tabular}{cl}  
    \begin{tabular}{c}
      \includegraphics[height=2cm, width=2cm]{images/sage-squared}
      \includegraphics[height=2cm, width=2cm]{images/python} \\
      \includegraphics[height=2cm, width=2cm]{images/cpp}
      \includegraphics[height=2cm, width=2cm]{images/mathematica} \\
      \includegraphics[height=1.5cm, width=3cm]{images/magma}
    \end{tabular}
    & \begin{tabular}{l}
        \parbox{0.5\linewidth}{%  change the parbox width as appropiate
          \begin{itemize}
          \item People often have a preferred software system. \pause
          \item Computations can be expensive. \pause
          \item Software changes often and requires maintenance. \pause
          \item Data is often more valuable than the software. \pause
          \item Verification of results is at most as computationally expensive. \pause
          \end{itemize}
        }
      \end{tabular}  \\
  \end{tabular}
\end{frame}

%%%%%%%%%%%%%%%%%%%%%%%%%%%%%%%%%%%%%%%%%%%%%%%%%%%%%%%%%%%%%%%%%%%%%%%%%%%%%%% 
%%%%%%%%%%%%%%%%%%%%%%%%%%%%%%%%%%%%%%%%%%%%%%%%%%%%%%%%%%%%%%%%%%%%%%%%%%%%%%% 
%%%%%%%%%%%%%%%%%%%%%%%%%%%%%%%%%%%%%%%%%%%%%%%%%%%%%%%%%%%%%%%%%%%%%%%%%%%%%%% 

\begin{frame}[fragile]{ History of File Formats}

  \begin{tabular}{cl}  
    \begin{tabular}{c}
      \includegraphics[width=.5\textwidth]{images/mps_file}
    \end{tabular}
    & \begin{tabular}{l}
      \parbox{0.4\linewidth}{%  change the parbox width as appropiate

      \begin{itemize}
      \item The LP file format and the MPS file format. IBM [1970s] (industry standards)
      \item \texttt{Mathematica} Notebooks. Wolfram Mathematica [1988]
      \item \texttt{OpenMath} (tree structure). Mike Dewar [2000]
      \item \texttt{IPython 0.12} Interactive Browser Notebooks (Jupyter) [2011]
      \item \texttt{polymake} File Format. E. Gawrilow, S. Hampe, and M. Joswig [2016]
      \end{itemize}
      }
    \end{tabular}  \\
  \end{tabular}
\end{frame}

%%%%%%%%%%%%%%%%%%%%%%%%%%%%%%%%%%%%%%%%%%%%%%%%%%%%%%%%%%%%%%%%%%%%%%%%%%%%%%% 
%%%%%%%%%%%%%%%%%%%%%%%%%%%%%%%%%%%%%%%%%%%%%%%%%%%%%%%%%%%%%%%%%%%%%%%%%%%%%%% 
%%%%%%%%%%%%%%%%%%%%%%%%%%%%%%%%%%%%%%%%%%%%%%%%%%%%%%%%%%%%%%%%%%%%%%%%%%%%%%% 

\begin{frame}[fragile]{Storing Computer Algebra Data}
  \begin{itemize}
  \item It's common to have multiple perspectives on an object in mathematics. 
  \item While storing mathematical data a choice of perspective must me made.
  \item Such a choice might not be describeable in an email.
  \end{itemize}
    
  Say we want to store:
  \begin{align*}
    p = 2y^3z^4 + (\bold{a} + 3)z^2 + 5\bold{a}y + 1
  \end{align*} \pause

  \begin{itemize}
  \item Some technicalities with the coefficients.
  \item Is y considered a coefficient of z?
  \item What is $\bold{a}$?
  \item How can we guarantee the objects behave as expected on load?
  \end{itemize}
\end{frame}

%%%%%%%%%%%%%%%%%%%%%%%%%%%%%%%%%%%%%%%%%%%%%%%%%%%%%%%%%%%%%%%%%%%%%%%%%%%%%%% 
%%%%%%%%%%%%%%%%%%%%%%%%%%%%%%%%%%%%%%%%%%%%%%%%%%%%%%%%%%%%%%%%%%%%%%%%%%%%%%% 
%%%%%%%%%%%%%%%%%%%%%%%%%%%%%%%%%%%%%%%%%%%%%%%%%%%%%%%%%%%%%%%%%%%%%%%%%%%%%%% 

\begin{frame}[fragile]{The \texttt{mrdi} File Format}
  \begin{tcolorbox}[]
    \begin{itemize}
    \item JSON file format.
    \item Similar to \texttt{polymake} format but generalizes to include algebraic data.
    \item Namespaces for semantic seperation.
    \item References stored with UUIDs.
    \item Prototype developed using \OSCAR.
    \end{itemize}
  \end{tcolorbox}
\end{frame}

\begin{frame}[fragile]{ Demo}
  \includegraphics[height=0.8\textheight]{images/demo}
\end{frame}

%%%%%%%%%%%%%%%%%%%%%%%%%%%%%%%%%%%%%%%%%%%%%%%%%%%%%%%%%%%%%%%%%%%%%%%%%%%%%%% 
%%%%%%%%%%%%%%%%%%%%%%%%%%%%%%%%%%%%%%%%%%%%%%%%%%%%%%%%%%%%%%%%%%%%%%%%%%%%%%% 
%%%%%%%%%%%%%%%%%%%%%%%%%%%%%%%%%%%%%%%%%%%%%%%%%%%%%%%%%%%%%%%%%%%%%%%%%%%%%%% 

\begin{frame}[fragile]{Tree Structure}
  \vspace*{-0.5cm} % Adjust this value as needed to reduce whitespace
  \begin{tabular}{cl}  
    \begin{tabular}{c}
      \includegraphics[width=0.43\textwidth, height=0.8\textheight]{images/tree-coloured}
    \end{tabular}
    & \begin{tabular}{l}
      \parbox{0.25\linewidth}{%  change the parbox width as appropiate
      \begin{align*}
        \\
        \\
        \\
        & 2y^3z^4 \\
        \\
        \\
        &  (a + 3)z^2 \\
        \\
        \\
        & 5ay \\
        \\
        \\
        & 1
      \end{align*}
      }
    \end{tabular}  \\
  \end{tabular}
\end{frame}

%%%%%%%%%%%%%%%%%%%%%%%%%%%%%%%%%%%%%%%%%%%%%%%%%%%%%%%%%%%%%%%%%%%%%%%%%%%%%%% 
%%%%%%%%%%%%%%%%%%%%%%%%%%%%%%%%%%%%%%%%%%%%%%%%%%%%%%%%%%%%%%%%%%%%%%%%%%%%%%% 
%%%%%%%%%%%%%%%%%%%%%%%%%%%%%%%%%%%%%%%%%%%%%%%%%%%%%%%%%%%%%%%%%%%%%%%%%%%%%%% 

\begin{frame}[fragile]{Example File Serialized with \OSCAR}
  \includegraphics[height=0.9\textheight]{images/polynomial-example}
\end{frame}

%%%%%%%%%%%%%%%%%%%%%%%%%%%%%%%%%%%%%%%%%%%%%%%%%%%%%%%%%%%%%%%%%%%%%%%%%%%%%%%
%%%%%%%%%%%%%%%%%%%%%%%%%%%%%%%%%%%%%%%%%%%%%%%%%%%%%%%%%%%%%%%%%%%%%%%%%%%%%%%
%%%%%%%%%%%%%%%%%%%%%%%%%%%%%%%%%%%%%%%%%%%%%%%%%%%%%%%%%%%%%%%%%%%%%%%%%%%%%%%

\begin{frame}[fragile]{Namespaces}
  \begin{itemize}
  \item We make no attempt to formalise the language of Mathematical Research Data.
  \item We rely on namespaces to pin the semantics to a specific software.
  \item Open source software provides example implementations for saving and loading.
  \item Freedom to upgrade encoding.
  \item Similar with PDF.
  \end{itemize}
\end{frame}

%%%%%%%%%%%%%%%%%%%%%%%%%%%%%%%%%%%%%%%%%%%%%%%%%%%%%%%%%%%%%%%%%%%%%%%%%%%%%%%
%%%%%%%%%%%%%%%%%%%%%%%%%%%%%%%%%%%%%%%%%%%%%%%%%%%%%%%%%%%%%%%%%%%%%%%%%%%%%%%
%%%%%%%%%%%%%%%%%%%%%%%%%%%%%%%%%%%%%%%%%%%%%%%%%%%%%%%%%%%%%%%%%%%%%%%%%%%%%%%

\begin{frame}[fragile]{ Serialized Toric Divisor}
  \includegraphics[width=\textwidth]{images/toric-divisor}
\end{frame}

%%%%%%%%%%%%%%%%%%%%%%%%%%%%%%%%%%%%%%%%%%%%%%%%%%%%%%%%%%%%%%%%%%%%%%%%%%%%%%%
%%%%%%%%%%%%%%%%%%%%%%%%%%%%%%%%%%%%%%%%%%%%%%%%%%%%%%%%%%%%%%%%%%%%%%%%%%%%%%%
%%%%%%%%%%%%%%%%%%%%%%%%%%%%%%%%%%%%%%%%%%%%%%%%%%%%%%%%%%%%%%%%%%%%%%%%%%%%%%%

\begin{frame}[fragile]{Schema}
    \begin{tabular}{cl}  
      \begin{tabular}{c}
        \begin{figure}
          \begin{minipage}{0.4\textwidth}
            \includegraphics[height=4cm, width=5cm]{images/blocks}
            \caption{\url{https://www.pexels.com/photo/plastic-shape-shorter-toy-11030155/}}
          \end{minipage}
        \end{figure}
    \end{tabular}
    & \begin{tabular}{l}
        \parbox{0.5\linewidth}{%  change the parbox width as appropiate
          \begin{itemize}
          \item A schema defines a structure for data. \pause
          \item Schema languages. (RELAX NG [2002], JSON Schema [2022])\pause
          \item Is possible to define recursive structure. \pause
          \item Schemata allow data to be validated before loading. \pause
          \item Adds structure to document based databases. \pause
          \item \polydb, Paffenholz [2017]
          \end{itemize}
        }
      \end{tabular}  \\
  \end{tabular}
\end{frame}

%%%%%%%%%%%%%%%%%%%%%%%%%%%%%%%%%%%%%%%%%%%%%%%%%%%%%%%%%%%%%%%%%%%%%%%%%%%%%%% 
%%%%%%%%%%%%%%%%%%%%%%%%%%%%%%%%%%%%%%%%%%%%%%%%%%%%%%%%%%%%%%%%%%%%%%%%%%%%%%%
%%%%%%%%%%%%%%%%%%%%%%%%%%%%%%%%%%%%%%%%%%%%%%%%%%%%%%%%%%%%%%%%%%%%%%%%%%%%%%%

\begin{frame}[fragile]{Data Collections}
  \begin{itemize}
  \item Johnson Solids.
  \item Non-general type surfaces in $\PP^4$.
  \item F-Theory models.
  \item Small Phylogenetic Trees. (Luis David Garcia-Puente, Jacob Porter)
  \end{itemize}
\end{frame}

%%%%%%%%%%%%%%%%%%%%%%%%%%%%%%%%%%%%%%%%%%%%%%%%%%%%%%%%%%%%%%%%%%%%%%%%%%%%%%% 
%%%%%%%%%%%%%%%%%%%%%%%%%%%%%%%%%%%%%%%%%%%%%%%%%%%%%%%%%%%%%%%%%%%%%%%%%%%%%%%
%%%%%%%%%%%%%%%%%%%%%%%%%%%%%%%%%%%%%%%%%%%%%%%%%%%%%%%%%%%%%%%%%%%%%%%%%%%%%%%

\begin{frame}[fragile]{}
  \begin{center}
    Thank You!
  \end{center}
\end{frame}

%%%%%%%%%%%%%%%%%%%%%%%%%%%%%%%%%%%%%%%%%%%%%%%%%%%%%%%%%%%%%%%%%%%%%%%%%%%%%%%
%%%%%%%%%%%%%%%%%%%%%%%%%%%%%%%%%%%%%%%%%%%%%%%%%%%%%%%%%%%%%%%%%%%%%%%%%%%%%%%
%%%%%%%%%%%%%%%%%%%%%%%%%%%%%%%%%%%%%%%%%%%%%%%%%%%%%%%%%%%%%%%%%%%%%%%%%%%%%%%

\end{document}
